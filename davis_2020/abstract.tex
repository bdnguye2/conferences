%\documentclass[journal=jctcce,manuscript=article]{achemso}
%\setkeys{acs}{articletitle=true}
%\setkeys{acs}{email=false}
\documentclass[12pt]{article}
\usepackage[T1]{fontenc} % Use modern font encodings
\usepackage{amsmath}
\usepackage{sectsty}
\usepackage{graphicx}
\usepackage{lastpage}
\usepackage{float}
\usepackage{fancyhdr}
\usepackage{fnpos}
\usepackage[english]{babel}
\usepackage{array}
\usepackage{droidsans}
\usepackage{charter}
\usepackage{setspace}
\usepackage[compact]{titlesec}
\usepackage{hyperref}
\usepackage{braket}


\usepackage[margin=1in]{geometry}



\title{Divergence of Many-Body Perturbation Theory in Noncovalent Interactions}
\author{\underline{Brian D. Nguyen},
  Guo P. Chen, Matthew M. Agee, \\ Asbj{\"o}rn M. Burow,
  Matthew P. Tang,
  Filipp Furche}

\begin{document}
\maketitle

\begin{abstract}
  Many-body perturbation theory (MBPT) has been the method of choice to
  predict noncovalent interactions (NIs). The assumption that ``weak''
  closed-shell interactions between distant electron pairs are accurately
  captured by MBPT is implicit in many applications as well as theoretical
  approaches such as local correlation methods. However, recent benchmark
  calculations for supramolecular complexes with $4-206$ atoms 
  
  Noncovalent interactions (NIs) play a large role in structural biology
  and supramolecular chemistry. The prediction of NIs remains an outstanding
  computational challenge. Many-body perturbation theory (MBPT) has been the
  approach to predict NIs such as the efficient second-order M{\o}ller-Plesset
  perturbation theory (MP2) which is accurate for small complexes
  based on the S66 benchmark. However, recent reports revealed large errors
  in NI energies of supramolecular complexes obtained from MBPT.
  Prompted by these errors, we compare the performance of MP2, spin-scaled MP2,
  dispersion-corrected semilocal density functional approximations (DFAs),
  and the post-Kohn--Sham random phase approximation (RPA) for predicting binding
  energies of complexes within the S66, L7, and S30L benchmarks. Numerical
  results demonstrated that the accuracy of MP2 severely deteriorates as
  the system size grows with an error rate of $0.1\%$ per electron.
  Whereas, empirical dispersion-corrected DFAs and RPA errors remain virtually
  constant.

  To analyze these results, the asymptotic adiabatic connection
  symmetry-adapted perturbation theory (AC-SAPT) is developed which uses
  monomers at full coupling whose ground-state density is constrained to
  the ground-state density of the complex. Taylor series expansion of AC-SAPT
  interaction energies with respect to the coupling strength integrand is
  shown to be convergent for nondegenerate monomers when RPA is used, while it
  spuriously diverges for second-order MBPT. Based on the analysis and numerical
  results, MBPT for NIs may safely be replaced with non-perturbative approaches
  such as RPA or coupled cluster methods.

\end{abstract}

\end{document}
