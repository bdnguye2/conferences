\documentclass[12pt]{article}
\usepackage[T1]{fontenc} % Use modern font encodings
\usepackage{amsmath}
\usepackage{sectsty}
\usepackage{graphicx}
\usepackage{lastpage}
\usepackage{float}
\usepackage{fancyhdr}
\usepackage{fnpos}
\usepackage{array}
\usepackage{droidsans}
\usepackage{charter}
\usepackage{setspace}
\usepackage[compact]{titlesec}
\usepackage{hyperref}
\usepackage{braket}
\usepackage{siunitx}
\usepackage[margin=1in]{geometry}



\title{Divergence of Many-Body Perturbation Theory in Noncovalent Interactions}
\author{\underline{Brian D. Nguyen},
  Guo P. Chen, Matthew M. Agee, \\ Asbj{\"o}rn M. Burow,
  Matthew P. Tang,
  Filipp Furche}

\begin{document}
\maketitle

\begin{abstract}
  Many-body perturbation theory (MBPT) has been the method of choice to
  predict noncovalent interactions (NIs). The assumption that ``weak''
  closed-shell interactions between distant electron pairs are accurately
  captured by MBPT is implicit in many applications as well as theoretical
  approaches such as local correlation methods. However, recent benchmark
  calculations for supramolecular complexes with $4-206$ atoms show that the
  accuracy of the second-order M{\o}ller-Plesset MBPT severely deteriorates
  as the system size grows with an error rate of approximately 0.1$\%$
  per valence electron. To analyze these unexpected results, an asymptotic
  adiabatic connection symmetry-adapted perturbation theory (AC-SAPT) is presented
  which uses monomers at full coupling. A nonperturbative ``screened
  second-order'' expression for the dispersion energy in terms of monomer
  quantities is derived using the fluctuation--dissipation theorem. Explicit
  expressions for the convergence radius of the AC-SAPT series are derived
  within the random phase approximation (RPA) and MBPT. Expansion of the
  AC-SAPT series within RPA for nondegenerate monomers is shown to always be
  convergent whereas, it is found to spuriously diverge for second-order MBPT,
  except for the smallest and least polarizable monomers. I will argue that
  the the divergence of MBPT for presumably ``weak'' dispersion interactions is
  caused by missing or incomplete ``electrodynamic'' screening of the Coulomb
  interaction due to induced particle--hole pairs between electrons in different
  monomers, leaving the effective interaction too strong for AC-SAPT to converge
  within MBPT even in moderately polarizable molecules with a few tens of atoms.
  Conclusions for electronic structure theory and computational practice will be
  discussed.
\end{abstract}

\end{document}
