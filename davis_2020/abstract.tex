%\documentclass[journal=jctcce,manuscript=article]{achemso}
%\setkeys{acs}{articletitle=true}
%\setkeys{acs}{email=false}
\documentclass[12pt]{article}
\usepackage[T1]{fontenc} % Use modern font encodings
\usepackage{amsmath}
\usepackage{sectsty}
\usepackage{graphicx}
\usepackage{lastpage}
\usepackage{float}
\usepackage{fancyhdr}
\usepackage{fnpos}
\usepackage[english]{babel}
\usepackage{array}
\usepackage{droidsans}
\usepackage{charter}
\usepackage{setspace}
\usepackage[compact]{titlesec}
\usepackage{hyperref}
\usepackage{braket}


\usepackage[margin=1in]{geometry}



\title{Divergence of Many-Body Perturbation Theory in Noncovalent Interactions}
\author{\underline{Brian D. Nguyen},
  Guo Chen, Matthew M. Agee, \\ Asbj{\"o}rn M. Burow,
  Matthew P. Tang,
  Filipp Furche}

\begin{document}
\maketitle

\begin{abstract}
  Noncovalent interactions (NIs) play a large role in structural biology
  and supramolecular chemistry. The prediction of NIs remains an outstanding
  computational challenge. Many-body perturbation theory (MBPT) has been the
  approach to predict NIs such as the efficient second-order M{\o}ller-Plesset
  perturbation theroy (MP2) which is accurate for small complexes
  based on the S66 benchmark. However, recent reports revealed large errors
  in NI energies of supramolecular complexes obtained from MBPT.
  Prompted by these errors, we compare the performance of MP2, spin-scaled MP2,
  dispersion-corrected semilocal density functional approximation (DFA),
  and the post-Kohn--Sham random phase approximation for predicting binding
  energies of complexes within the S66, L7, and S30L benchmarks. Numerical
  results demonstrated that the accuracy of MP2 severely deteriorates as
  the system size grows with an error rate of $0.1\%$ per electron.
  Whereas, empirical dispersion-corrected DFAs and RPA errors remain virtually
  constant.

  To analyze these results, the asymptotic adiabatic connection
  symmetry-adapted perturbation theory (AC-SAPT) is developed which uses
  monomers at full coupling whose ground-state density is constrained to
  the ground-state density of the complex. Expansion of AC-SAPT interaction energies
  from Taylor expansion of the coupling strength integrand is shown to be convergent
  for nondegenerate monomers when RPA is used, while it spuriously diverge for
  second-order MBPT. Based on the analysis and numerical results, MBPT for
  NIs may safely be replaced with non-perturbative approaches such as RPA
  or coupled cluster methods.
%  This material is based upon work supported by the National Science Foundation 
%  under CHE-1464828 and CHE-1800431  

\end{abstract}

\end{document}
