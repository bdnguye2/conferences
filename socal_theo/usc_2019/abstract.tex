\documentclass[journal=jctcce,manuscript=article]{achemso}
\setkeys{acs}{articletitle=true}
\setkeys{acs}{email=false}

\usepackage{chemformula} % Formula subscripts using \ch{}
\usepackage[T1]{fontenc} % Use modern font encodings
\usepackage{amsmath}
\usepackage{sectsty}
\usepackage{graphicx}
\usepackage{lastpage}
\usepackage{float}
\usepackage{fancyhdr}
\usepackage{fnpos}
\usepackage[english]{babel}
\usepackage{array}
\usepackage{droidsans}
\usepackage{charter}
\usepackage{setspace}
\usepackage[compact]{titlesec}
\usepackage{hyperref}
\usepackage{braket}
\usepackage{epstopdf}
\usepackage{accents}
\usepackage{footnote}
\usepackage{threeparttable}

\title{Divergence of Perturbation Theory in Noncovalent Interactions}
\author{\underline{Brian Nguyen}, Guo Chen, Matthew M. Agee, Asbj{\"o}rn M. Burow,
Filipp Furche}

\begin{document}
\vspace*{-0.5in}
\begin{abstract}
  Noncovalent interactions (NIs) play a large role in structural biology
  and supramolecular chemistry. The prediction of NIs remains an outstanding
  computational challenge. Couple-cluster singles, doubles, and perturbative
  triples (CCSD(T)) has been the method to predict NIs. However, CCSD(T) is
  limited to the computation of small molecules which has limited the
  understanding of NIs to small complexes of up to $\sim 50$ atoms. Electronic
  structure methods were developed based on small molecules with the
  underlying assumptions that these methods may be scaled up to larger ones
  without deterioration in accuracy. Perturbation theory has been the common
  approach to predict NIs. The efficient second-order M{\o}ller-Plesset
  perturbation theroy (MP2) has been shown to be accurate for small complexes
  based on the S22 benchmark. In this work, we demonstrated that
  the accuracy of MP2 severely deteriorates as the system size grows and
  it is indicative of the limitations of perturbation theory for NIs.

  The first step to accurately compute NIs requires methods to account for
  London dispersion interactions, which can account up to 200$\%$ of the interaction
  energies. RPA, which is based on the interactions between virtual
  ground-state density fluctuations, presents itself as an accurate
  way to describe NIs. It correctly captures three- and higher-body
  dispersion effects due to inclusion of polarization and screening.
  New numerical results of the S66, L7, S30L, and ROT34 test sets for RPA
  illustrate the importance of higher-body dispersion effects.
  This material is based upon work supported by the National Science Foundation 
  under CHE-1800431  

\end{abstract}

\end{document}
