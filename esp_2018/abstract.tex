%%%%%%%%%%%%%%%%%%%%%%%%%%%%%%%%%%%%%%%%%%%%%%%%%%%%%%%%%%%%%%%%%%%%%%%%%%%
% This is a LaTeX template for abstracts for WATOC 2017 in Munich, Germany
%
% Author instructions are below 
%
% Do not change the setup                                                    
%%%%%%%%%%%%%%%%%%%%%%%%%%%%%%%%%%%%%%%%%%%%%%%%%%%%%%%%%%%%%%%%%%%%%%%%%%%%%%%
\documentclass[12pt]{scrartcl}

%%%%%%%%%%%%%%%%
% Preamble     %
% DO NOT TOUCH %
%%%%%%%%%%%%%%%%
\usepackage[utf8]{inputenc}
\usepackage[T1]{fontenc}
\usepackage{graphicx}
\usepackage{txfonts}
%\usepackage[left=1.25in,right=1.25in,top=1.0in,bottom=2.10in]{geometry}
\usepackage{ifthen}
\usepackage{url}
%%%%%%%%%%%%%%%%%%%%%%%%%%%%%%%%%%%%%%%%%%%%%%%%%%%%%%%%%%
% Here you can load your own packages if needed. 
% 
% \usepackage{whatever}
%%%%%%%%%%%%%%%%%%%%%%%%%%%%%%%%%%%%%%%%%%%%%%%%%%%%%%%%%%

\pagestyle{empty}

%%%%%%%%%%%%%%%%%%%%%%%%%%%%%%%%%%%%%%%%%%%%%%%%%%%%%%%%%%
%
% DO NOT CHANGE ANY OF THE FOLLOWING DEFINITIONS
%
%%%%%%%%%%%%%%%%%%%%%%%%%%%%%%%%%%%%%%%%%%%%%%%%%%%%%%%%%%
\setlength{\parindent}{0cm}
\renewcommand\refname{}
\newcommand{\Title}[1]{
   \begin{center}
      \large{
         \textbf{
            #1
            \vspace{1.0\baselineskip}
         }
      }
   \end{center}
}
\newcount{\numauthors}
\numauthors=0%
\newcommand{\AddAuthor}[2]{%
   \advance\numauthors by 1%
   \large{%%
      \ifnum\numauthors=1%
         \textbf{#1}\textsuperscript{#2}%
      \else%
         \textbf{, #1}\textsuperscript{#2}%
      \fi%
   }%
}%
\newenvironment{Authors}{
   \begin{center}
}{
   \end{center}
}
\newenvironment{Address}{
   \begin{center}
}{
   \end{center}
}
\newenvironment{Abstract}{
   {~}
   \vspace{
      \baselineskip
   }
   \newline{}
   \normalsize
}{
}

%%%%%%%%%%%%%%%%%%%%%
% Beginning of Abstract 
%%%%%%%%%%%%%%%%%%%%%

% All formatting will be dealt with by the functions \Title, \AddAuthor, \AddAddress, ...
\begin{document}
% Your title goes here
\Title{Casimir-Polder size consistency: a case for RPA}

\begin{Authors}
   % For every author add an \AddAuthor command
   % Superscript for affiliation goes in as second argument.
   % leave empty if no affiliation
   \AddAuthor{\normalfont \normalsize Brian
     D. Nguyen, Matthew Agee, Guo Chen, Asbjörn M. Burow,
     Filipp Furche}{}%
\end{Authors}
%
\begin{Address}
   % Your address goes here. For every address add an \AddAddress command
   % Superscript is the second argument
   \AddAddress{University of California, Irvine, Department of
     Chemistry, 1102 Natural Sciences II, \\ Irvine, CA 92697-2025, USA}{}%
\end{Address}

\begin{Abstract}

For years, quantum chemical approaches have aimed to accurately and
efficiently predict noncovalent interactions (NIs) between atoms
and molecules. These include dispersion interactions, $\pi$-$\pi$
stacking, or hydrogen and halogen bonding that play an important
role in structural biology and supramolecular chemistry. Here we
assess the performance of the random phase approximation (RPA),
which has evolved from a semi-analytical technique for model
Hamiltonians to a powerful tool for ab initio electronic structure
calculations in chemistry and materials science. The accuracy of
RPA for weakly interacting systems from the L7, S66, and S30L benchmarks
illustrates the critical importance of beyond the pairwise additivity of
NIs in moderately large sized molecules with 100 - 200 atoms. Comparison
to the second order M{\o}ller-Plesset perturbation theory (MP2) and
semilocal density functional approximations (DFAs) with dispersion
correction reveals that RPA performs better than MP2 and on par with
dispersion corrected-DFAs.


% If you prefer using bibtex, please use "angew" as style

%%%%%%%%%%%%%%%%%%%%%%%%%%%
% End of Abstract         %
% No text after this line %
%%%%%%%%%%%%%%%%%%%%%%%%%%%
\end{Abstract}
\end{document}
